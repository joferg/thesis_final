Complex multi-scale atmospheric phenomena, like tropical 
cyclones, challenge conventional weather and climate models, 
which use relatively coarse uniform-grid resolutions to cope with 
computational costs. Adaptive Mesh Refinement (AMR) techniques 
mitigate these challenges by dynamically and transiently placing high-resolution grids 
over salient features, thus providing sufficient local resolution while 
limiting the computational burden. 
This thesis explores the development of AMR within a new 
non-hydrostatic, finite-volume dynamical core and demonstrates its 
effectiveness in improving model accuracy and its ability to resolve 
multi-scale features. The AMR dynamical core is implemented in a 
hierarchy of models of increasing complexity, from an idealized 2D 
shallow water configuration to the non-hydrostatic 3D equation set 
with subgrid-scale physics parameterization schemes. The research explores 
effective refinement practices and assesses the benefits achieved 
with increased dynamic refinement. It is shown that AMR is a 
powerful modeling approach that bridges the resolution gap for 
extreme weather events.

In Chapter \ref{chap:basicsw} we utilized a fourth-order
finite-volume model on a cubed-sphere grid, which is adaptive in both
space and time, to demonstrate the effectiveness of the AMR in resolving
and tracking chosen features of interest while maintaining large-scale
smooth flows.  Using selected shallow-water and advection test cases, we
evaluated the AMR's ability to track and resolve features of interest
without creating distortions or numerical noise in the large-scale
smooth flows at the interfaces between meshes. 
Static and dynamic refinements were analyzed to determine the 
strengths and weaknesses of AMR in both complex flows with small-scale
 features and large-scale smooth flows. The different test cases required 
 different AMR criteria, such as vorticity, height, or gradient-based thresholds, 
 in order to achieve the best accuracy. The simulations showed that the 
 model can accurately resolve key local features without requiring global 
 high-resolution grids. The adaptive grids are able to track features of interest 
 reliably without inducing noise or visible distortions at the coarse-fine interfaces. 
 Furthermore, the AMR grids keep any degradations of the large-scale smooth flows to a minimum.
 
In Chapter \ref{chap:forcedsw} we implemented two different forcing schemes designed to 
mimic the effects of moisture in the atmosphere within the 2D AMR
shallow water system. The first moist physics
framework added water vapor variable and models convection
as a mass sink triggered by saturation. We implemented a
strengthened vortex test case with this setup.  In the second
forcing framework, a more complex moisture representation
is used, consisting of vapor, cloud, and rain variables.
The effects of moisture were coupled to the momentum equations
through a potential temperature variable, linked to
the moisture variables through latent heat. This physics
scheme was used with a barotropic instability test case. 
With both forcing systems and test cases, we observe the evolution
 of features of interest at various resolutions and 
with different refinement strategies. 
We also investigated the AMR's effect on the physics 
forcing as grid resolutions changed.
These simulations can help develop AMR tagging strategies and refinement criteria.
Both sets of simulations showed that the starting resolution must be
able to adequately resolve the feature of interest to maximize
the effectiveness of AMR. AMR cannot remove the errors
caused by coarse grids before refinement begins. Additional refinement with
AMR beyond the base grid level did improve the model, especially
with regards to the small-scale vorticity features in the strengthening vortex test case.
To obtain such early refinement with AMR, the tagging
criteria must be tailored to properties that are uniquely associated
with the origins of the feature of interest. This is difficult even in these 
idealized shallow water systems.

In Chapter \ref{chap:3dmodel}, we implemented AMR in 
the non-hydrostatic finite-volume Chombo AMR dycore
and tested it for idealized 3D atmospheric flows on the sphere. 
We used two test cases: the dry dycore colliding modons
test and an idealized TC test with a simple moist physics parameterization scheme.
In the modon test, AMR functioned as expected. It was able to tag, refine, and follow the modons,
and the added resolution  effectively reduced global the vorticity errors.
The error convergence properties met our expectations given the numerical 
schemes used and the current status of the model. However, a
 few results triggered the need for additional in-depth examinations.
Several stability problems were observed in both test cases 
that need to be better understood and addressed.

In the idealized TC test, the AMR runs were able to effectively reproduce results from uniform high resolution runs.
The physics scheme was able to function effectively over multiple levels of refinement. No noise
or instabilities were observed at coarse-fine boundaries, though some
 numerical artifacts were observed on polar panel edges. 
Several aspects of the TC's evolution were
sensitive to the tagging criteria and AMR coverage such as the generation of the secondary vortex.
Having  AMR levels applied initially is the most effective way of improving results.
However, the delayed AMR tests demonstrated that the TC can still 
undergo the same strengthening processes as high resolution runs once refinement is triggered.
There is a narrow window of flexibility, in which the triggering of AMR allows the TC in the AMR 
run to catch up to the results in the high resolution uniform run.
The most promising refinement technique is a combination of some initial 
refinement and AMR. The initial refinement limits error growth at the 
base resolution and ensures that the model can resolve the feature of interest. 
Additional AMR level then enable the feature to then be fully resolved.
For example, in tracking and resolving TCs in a realistic climate simulation, 
a static region of refinement
could be placed over regions of cyclogenesis. Any storms that
develop could be further refined and followed with AMR as they traverse
and exit the region of static refinement.


\subsection{Collaborations}
This research has built collaborations between the University of Michigan (UofM), the 
Applied Numerical Algorithms Group (ANAG) at the Lawrence Berkeley National Laboratory (LBNL), and
the University of California, Davis (UC Davis). In particular, direct contributors to 
the research contained in this thesis include 
Christiane Jablonowski (UofM), Hans Johansen (LBNL), 
Peter McCorquodale (LBNL), Phillip Colella (LBNL), and Paul Ullrich (UC Davis).

\subsection{Future Work}

\textbf{AMR dycore development:}
Additional effort is required in implementing, validating, and improving AMR within GCMs.
Use of AMR in a 3D non-hydrostatic dycore is a relatively recent application. 
Further development includes the creation
of more complex models with realistic physics schemes, with
the goal of running aqua-planet simulations with full-physics schemes. 
Implementing vertical refinement with AMR is another algorithm development focus.
With an AMR non-hydrostatic model, grid resolutions on the order of 1km or finer are feasible.
At these resolutions, the aspect ratios between the horizontal and vertical lengths of a grid cell 
can be come highly skewed.  The addition of more vertical cells would alleviate this issue, and 
increase vertical resolution. 

 \vspace{6pt}
\textbf{Scale aware subgrid physics parameterizations:}
The effectiveness of VRGCMs is limited by the sub-grid physical parameterizations used to
approximate atmospheric features including radiation and sub-grid scale process like convection, 
clouds, and turbulence that cannot be resolved by the dycore.
In adaptive schemes, the subgrid-scale parameterizations need to be able to adjust for changes in scale.  
A model will need to be able to phase out certain subgrid-scale processes, like deep
convection, as resolution is increased and these processes are resolved on the actual mesh.
One novel research avenue is the application of artificial intelligence to aid in the development of
these scale-aware physics schemes.

\vspace{6pt}
\textbf{AMR and topography:}
Choosing an effective method to handle topography is a challenging endeavor for any model. 
The question of smoothing always arises as very steep topological gradients can cause issues 
for the vertical coordinates and create numerical noise.  AMR faces the additional hurdle of how to deal 
with topography when the grid is refined over it.  Literature on this subject is rather sparse, but there are 
several possible methods of approaching this issue.  One traditional method is to merely interpolate the
 existing coarse-grid topography to the new grid cells.  This method, though preserving monotonicity, will not necessarily 
 improve orographic representation as it cannot make mountains taller, valleys deeper, or gradients steeper.   
 A second method would be to actually alter the topographic features making them more pronounced 
 and even steeper or taller when the grid is refined.  The topography can then be smoothed down for 
 coarse grids and merely updated back to the high-resolution version when AMR refines it.  
 Such a method provides a more realistic orographic representation but could be numerically 
 unsound, causing air mass conservation issues and creating large gravity waves.  A third method 
 is to merely keep a static refinement mesh over large topographic features, though this 
 could be computationally inefficient.