Complex multi-scale atmospheric phenomena, like tropical 
cyclones, challenge conventional weather and climate models, 
which use relatively coarse uniform-grid resolutions to cope with 
computational costs. Adaptive Mesh Refinement (AMR) techniques 
mitigate these challenges by dynamically and transiently placing high-resolution grids 
over salient features, thus providing sufficient local resolution while 
limiting the computational burden. 

This thesis explores the development of AMR, a technique that 
has been featured only sporadically in the atmospheric science 
literature, within a new nonhydrostatic, 
finite-volume dynamical core and demonstrates AMR's effectiveness in 
improving model accuracy and ability to resolve multi-scale features. 
This high-order finite-volume model implements adaptive refinement in both space and 
time on a cubed-sphere grid using a mapped-multiblock mesh technique developed with
the Chombo AMR library. The AMR dynamical core is implemented in a hierarchy of
models of increasing complexity, from an idealized 2D shallow water configuration 
to the nonhydrostatic 3D equation set with subgrid-scale parameterizations schemes. 
AMR's numerical accuracy, computational efficiency, and ability to track and resolve
multifaceted and evolving features are assessed with
a variety of existing and new test cases, implemented within each model iteration.

Both static and dynamic refinements are analyzed to 
determine the strengths and weaknesses of AMR in both 
complex flows with small-scale features and large-scale smooth flows.
The different test cases required different AMR criteria, such 
as vorticity, or minimum pressure based thresholds, 
in order to achieve the best accuracy for cost. Simulations show 
that the model's AMR can accurately resolve key local features in 
both shallow water and 3D test cases without 
requiring global high-resolution grids, as the adaptive grids are able 
to track features of interest reliably without inducing noise or visible 
distortions at the coarse-fine interfaces. Furthermore, the AMR grids 
keep degradation of the large-scale smooth flows to a minimum.
2D and 3D physics parameterizations are able to 
function effectively over multiple levels of refinement, though
the parameterizations are sensitive to grid resolution.

AMR is most effective when refinement is triggered early 
or when the base uniform resolution can partially 
resolve the features of interests. Very coarse base resolutions
lead to large initial errors that cannot be overcome by AMR.
However, the addition of refinement later in the simulation still results
in significant improvements, especially in 
resolving small-scale features. The research showed that
flow properties, such as strong gradients or rainbands, 
can be sensitive to small changes in AMR
criteria. These may delay the onset of the refinement
 or alter the shape of the refined area, which
impacts the evolution of the flow. With coarse 
base resolutions, the tagging criteria must 
therefore be uniquely tailored to capture the early 
growth phases of the feature of interest.
A promising refinement technique is a combination of 
some initial refinement and AMR. 
The initial refinement limits error growth at the base 
resolution and ensures that the model can 
resolve the feature of interest. Overall,
AMR is shown to be a powerful modeling approach 
that bridges the resolution gap for extreme weather events.